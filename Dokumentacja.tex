\documentclass[11pt,a4paper]{article}
\usepackage{graphicx}
\usepackage{polski}
\usepackage[utf8]{inputenc}
\usepackage{tabularx}
\usepackage[hidelinks]{hyperref}
\usepackage{caption}
\usepackage{float}

\begin{document}
\begin{titlepage}
\begin{center}

\textsc{\Huge KATEDRA INFORMATYKI}\\[1.0cm]
\textsc{\LARGE Wydział Informatyki, Elektroniki i Telekomunikacji AGH}\\[1.0cm]
\includegraphics[width=0.4\textwidth]{./Obrazy/logo}~\\[1cm]

\textsc{\Large Testowanie dostępnych na rynku rozwiązań do analizy kryminalnej.}\\[1.0cm]

\emph Krzysztof \textsc{Trzepla} \\
\href{mailto:krzysztof.trzepla@gmail.com}{krzysztof.trzepla@gmial.com} \\
\emph Michał \textsc{Bigaj} \\ 
\href{mailto:mbigaj@gmail.com}{mbigaj@gmail.com} \\[0.5cm]
\textbf{Wersia 1.1}

\vfill
{\large \today}

\end{center}
\end{titlepage}


\tableofcontents
\newpage

\section{Sformułowanie zadania projektowego}

Celem niniejszego projektu jest poznanie oraz przeprowadzenie testów funkcjonalnych wybranych, dostępnych na rynku  aplikacji służącego do analizy kryminalnej. W ramach projektu przetestowane zostanie następujące oprogramowanie:
\begin{itemize}
	\item Analyst’s Notebook
	\item ArcGIS
	\item Maltego
	\item Palantir
	\item CrimeStat
\end{itemize}
Każda z wymienionych aplikacji będzie przeanalizowana i oceniona według następujących kryteriów:
\begin{enumerate}
	\item Zakres dostępnych funkcji
	\item Intuicyjność interfejsu oraz łatwość obsługi
	\item Aktualny stan prac nad produktem
	\item Sposób licencjonowania oraz cena produktu
	\item Parametry techniczne i technologiczne
\end{enumerate}

\section{Analiza wybranego oprogramowania} 

\subsection{Analyst's Notebook}

Analyst’s Notebook służy do graficznej prezentacji wiedzy o obiektach i powiązaniach pomiędzy obiektami w formie diagramów oraz do analizy zgromadzonych informacji. W zależności od charakteru informacji, obiekty są prezentowane na diagramie w postaci ikon oraz innych symboli takich jak linie tematów, ramki zdarzeń, prostokąty, koła, bloki tekstu, obiekty OLE.

\subsubsection*{Zakres dostępnych funkcji}

\begin{itemize}
	\item Tworzenie diagramów powiązań wykorzystywanych do analizy związków pomiędzy obiektami, które umożliwiają lepsze zrozumienie struktury wzajemnych relacji. Są również pomocne w analizach dotyczących przepływu towarów, a także ułatwiają identyfikację kluczowych osób, ich wspólników, metod działania czy sposobu komunikacji.
	\item Tworzenie diagramów chronologicznych mogących służyć do przedstawiania zdarzeń oraz ich powiązań z innymi obiektami w ujęciu czasowym. Są pomocne przy analizie przyczynowo-skutkowej dotyczącej np. działalności grupy osób.
	\item Tworzenie diagramów mieszanych, posiadających cechy omówionych wyżej rodzajów diagramów.
	\item Import danych z zewnętrznych źródeł w formacie \emph{txt}, \emph{csv}, \emph{tsv}, \emph{xml} oraz ze schowka \emph{Windows}.
	\item Wyszukiwanie graficzne
	\item Wyszukiwanie tekstowe
	\item Wyszukiwanie pośrednich powiązań pomiędzy obiektami odległymi
	\item Wyszukiwanie klastrów
	\item Wyszukiwanie obiektów podobnych
	\item Zmianę układu diagramu
	\item Umieszczanie elementów diagramu w tle
	\item Tworzenie raportów.
\end{itemize}

\subsubsection*{Intuicyjność interfejsu oraz łatwość obsługi}

Za pomocą prostego w obsłudze okna dialogowego, użytkownik określa sposób interpretacji danych wejściowych (podział na rekordy i pola rekordów) oraz przetwarzania ich na obiekty, połączenia, atrybuty i karty informacyjne. Użytkownik decyduje o sposobie wypełniania właściwości obiektów (takich jak identyfikator, etykieta itd.) i połączenia poprzez przypisanie im poszczególnych pól rekordu. Obsługa programu jest intuicyjna, choć szata graficzna pozostawia wiele do życzenia.

\subsubsection*{Stan prac na produktem}

Aktualnie produkt jest rozwijany przez firmę \emph{IBM}. Ostatnia aktualizacja oprogramowania do wersji \emph{8.9.7} miała miejsce 5 lutego 2014 roku.

\subsubsection*{Sposób licencjonowania oraz cena produktu}

Oprogramowanie Analyst's Notebook jest chronione Międzynarodową Umową Licencyjną na Program (\emph{IPLA}), która ogranicza Licencjobiorcy prawa do używania programu. Cena produktu waha się i wynosi od 7000\$ do 19000\$ za roczną licencję na jeden komputer.

\subsubsection*{Parametry techniczne i technologiczne}

Wymagania sprzętowe:
\begin{itemize}
	\item Minimalne:
		\begin{itemize}
			\item procesor taktowany częstotliwością 1.4 GHz
    		\item 512 MB pamięci RAM
    		\item 1.5 GB wolnej przestrzeni dyskowej
		\end{itemize}
	\item Zalecane: 
		\begin{itemize}
			\item procesor taktowany częstotliwością co najmniej 2 GHz
    		\item co najmniej 2 GB pamięci RAM
    		\item 2 GB wolnej przestrzeni dyskowej
		\end{itemize}
\end{itemize}

\raggedright{Wymagane oprogramowanie:}
\begin{itemize}
	\item System operacyjny: Windows 8 Professional/Enterprise, Windows 7 Professional/Enterprise/Ultimate, Microsoft Vista Business/Enterprise/Ultimate SP1 lub późniejszy, Microsoft Windows XP Professional SP3 lub późniejszy
	\item Microsoft .NET Framework 3.5 SP1
	\item Microsoft Windows Installer 4.5
	\item przeglądarka plików PDF
	\item Google Earth 5.0 lub późniejszy
	\item Microsoft Excel
	\item Microsoft PowerPoint
\end{itemize}

\subsection{ArcGIS}

Obejmuje pakiet programów, który jest wykorzystywwany do Systamów informacji Geograficznej (GIS). W jego skład wchodzą: ArcReader, ArcViewer(ArcGIS for Desktop Basic), ArcEditor(ArcGIS for Desktop Standard),ArcInfo(ArcGIS for Desktop Advanced). Ponad to dostępne są również produkty na platformy mobline  ArcGIS Mobile i ArcPad.
ArcGIS Mobile to środowisko deweloperskie do tworzenia aplikacji moblinych.

\subsubsection*{Zakres dostępnych funkcji}

\begin{itemize}

	\item ArcReader, darmowy program służący do podstawowego podglądu map i danych GIS, stworzonych przy użyciu ArcGIS Publisher.
	\item ArcViewer(ArcGIS for Desktop Basic) umożliwa podgląd i edycję danych GIS przechowywanych w plikach oraz podgląd danych trzymanych w relacyjnej bazie danych.
	\item ArcEditor(ArcGIS for Desktop Standard) posiada funkcje ArcViewer, ponadto umożliwa tworznie map, podstawową analizę danych przestrzennych, także w trybie współbieżnym, edycję danych rastrowych, wektoryzację, edycję danych wektorowych, zarządzanie pokryciami, edycję sieci geometrycznych.
	\item ArcInfo(ArcGIS for Desktop Advanced) oprócz funkcjonalności ArcEditora zwiększa możliwości analizy danych przestrzennych, geoprzetwarzania i zarządzania danymi.
\end{itemize}

\subsubsection*{Intuicyjność interfejsu oraz łatwość obsługi}

Graficzny interface zapewnia wygodne i sprawne poruszanie się po aplikacji. Brak dodatkowych efektów graficznych sprawia że program jest bardziej przejrzysty.

\subsubsection*{Stan prac na produktem}

 Aktualna wersja to 10.2. Została wydana w czerwcu 2013.

\subsubsection*{Sposób licencjonowania oraz cena produktu}

 Oprogramowanie jest objęte licencją oprogramowania zamkniętego (proprietary software). Dostępny jest również 30 dniowy trial. Podane ceny mogą się wahać w zależności od miejsca złożenia zamówienia i ilości licencji oraz możliwości współbieżnej pracy.

\centering
	\begin{tabular}{ | p{6cm}| p{6cm} | }
    	\hline
    	\textbf{Nazwa produktu} & \textbf{Cena}\\ \hline
	ArcGIS 10.2.1 do użytku domowego &  100 \$ \\ \hline
    	ArcGIS 10.2.1 for Desktop Basic & 2857 \$ \\ \hline
    	ArcGIS 10.2.1 for Desktop Standard & 5713 \$ \\ \hline
    	ArcGIS 10.2.1 for Desktop Basic & 7295 \$ \\ \hline
  	\end{tabular}
\raggedright
\subsubsection*{Parametry techniczne i technologiczne}

\begin{itemize}
	\item Minimalne:
		\begin{itemize}
		\item CPU 2.2 GHZ
    		\item 2 GB RAM
		\item wyświtlacz 24-bitowa głębia kolorów
		\item rozdzielczość ekranu 1024 x 768
		\item 2.4GB miejsca na dysku
		\item karta graficzna z 64MB RAM 
		\item OpenGL 2.0
		\item Połączenie z siecią
		\end{itemize}
	\item Zalecane:
		\begin{itemize}
		\item Intel Pentium 4, Intel Core Duo, lub Xeon 
    		\item karta graficzna z 256MB RAM lub więcej i wsparciem 24-bitowej głębii kolorów
    		\item Shader Model 3.0 lub wyższy
		\end{itemize}
\raggedright{Wymagane oprogramowanie:}		
	\item Desktop:	
		\begin{itemize}
		\item Windows XP SP2 i późniejsze
    		\item Windows Server 2003 SP2 i póżniejsze
    		\item CPU 
		\end{itemize}
	\item Server(tylko x64): 
		\begin{itemize}
		\item Wsparcie dal RHEL 5 i późniejsze
    		\item Wsparcie dal SLES 11 i późniejsze
    		\item dostęp do internetu 1Mb+
    		\item wyświetlacz o rozdzielczości 1920x1080
		\end{itemize}
	\item Mobile: 
		\begin{itemize}
		\item iOS 3.1.2 i późniejsze
    		\item Android 2.2 i późniejsze
    		\item Windows Phone 7 i późniejsze
    		\item Windows Mobile 6 i późniejsze 
		\end{itemize}
\end{itemize}
\raggedright
\subsection{Maltego}

Maltego to platforma mająca na celu wykrycie zagrożeń występujących w środowisku danej organizacji. Aplikacja umożliwia lokalizację zarówno pojedynczych punktów zapalnych, jak i sieci zależności pomiędzy krytycznymi infrastrukturami. Agregacja oraz wizualizacja zbieranych danych w znacznym stopniu ułatwia prowadzenie technik śledczych. 

\subsubsection*{Zakres dostępnych funkcji}

\begin{itemize}
	\item Zbieranie oraz wykrywanie powiązań pomiędzy grupami osób, firm oraz organizacji na podstawie \textit{białego wywiadu}, to jest na podstawie informacji pochodzących z ogólnie dostępnych źródeł
	\item Analiza sieci internetowej oraz dokumentów i plików pod kątem wybranych słów kluczowych
	\item Wyszukiwanie graficzne
	\item Wyszukiwanie tekstowe
	\item Tworzenie raportów.
\end{itemize}

\subsubsection*{Intuicyjność interfejsu oraz łatwość obsługi}

Dzięki wbudowanemu interfejsowi graficznemu prezentacja zgromadzonych wyników jest bardzo czytelna. Mnogość dostępnych funkcji pozwala w dużym stopniu na dostosowanie aplikacji do własnych potrzeb. Bogata szata graficzna czyni z pracy z tym narzędziem przyjemność.

\subsubsection*{Stan prac na produktem}

Produkt został stworzony przez firmę \emph{Paterva}. Ostatnia aktualizacja oprogramowania do wersji \emph{3.0} miała miejsce w styczniu 2011 roku.

\subsubsection*{Sposób licencjonowania oraz cena produktu}

Oprogramowanie Maltego jest chronione licencją typu \emph{one-site}, która zezwala na instalację programu wyłącznie na jednym komputerze. Produkt dostępny jest w dwóch wersjach: Community Edition (darmowa) oraz Commercial Edition (płatna).\\

\begin{table}[H]
	\centering
	\begin{tabular}{ | p{6cm}| p{6cm} | }
    	\hline
    	\textbf{Community Edition} & \textbf{Commercial Edition}\\ \hline
    	Nie dla użytku komercyjnego & Dla użytku komercyjnego \\ \hline
    	Maksymalnie 12 wyników per transformacja & Brak limitu na ilość wyników per transformacja \\ \hline
    	Nieszyfrowana komunikacja pomiędzy klientem a serwerem & Komunikacja pomiędzy klientem i serwerem zabezpieczona protokołem SSL \\ \hline
    	Mało wydajny serwer & Najwydajniejszy serwer \\ \hline
    	Aktualizacje dostępne tylko przy zmianie głównej wersji & Aktualizacje dostarczane na bieżąco \\ \hline
    	Brak wsparcia technicznego & Pełne wsparcie techniczne \\
    	\hline
  	\end{tabular}
  	\caption*{Najważniejsze różnice pomiędzy dostępnymi wersjami}
\end{table}

\subsubsection*{Parametry techniczne i technologiczne}

Wymagania sprzętowe:
\begin{itemize}
	\item Minimalne:
		\begin{itemize}
			\item procesor taktowany częstotliwością 2 GHz
    		\item 2 GB pamięci RAM
    		\item dostęp do internetu 64Kb
    		\item wyświetlacz o rozdzielczości 1024x768
		\end{itemize}
	\item Zalecane: 
		\begin{itemize}
			\item procesor firmy Intel z serii \emph{i7}
    		\item 8 GB pamięci RAM
    		\item dostęp do internetu 1Mb+
    		\item wyświetlacz o rozdzielczości 1920x1080
		\end{itemize}
\end{itemize}

\raggedright{Wymagane oprogramowanie:}
\begin{itemize}
	\item System operacyjny:  Windows XP/Vista/7 oraz wiele dystrybucji Linux
	\item Java 1.6 lub wyższa
\end{itemize}

\subsection{Palantir Gotham}


Służy do integracji danych ustrukturyzowanych i nieustrukturyzowanych. Zapewnia zaawansowany system przeszukiwania oraz odkrywania zależności. Umożliwia zarządzanie wiedzą, oraz ułatwia bezpieczną współpracę.

\subsubsection*{Zakres dostępnych funkcji}

\begin{itemize}
	\item Kategoryzacja badanego świata
	\item Jednoczesne przeszukiwanie wielu baz danych
	\item Umożliwia współbieżne opracowywanie danych
	\item Rozbudowany system logowania
	\item Umożliwia pracę offline
	\item Zapwnia bezpieczną pracę nad poufnymi danymi
\end{itemize}

\subsubsection*{Intuicyjność interfejsu oraz łatwość obsługi}

Główne okno aplikacji składa się z obszaru roboczego i paska zadań. Obszar roboczy może zawierać mapy z rozmieszczonymi danymi lub grafy zależności pomiędzy różnymi obiektami, dodatkowo umożliwia wstawianie i zaawansowane manipulowanie danymi przez intuicyjne menu wyświetlające się w około kursora.  

\subsubsection*{Stan prac na produktem}
Produkt jest rozwijany przez Palantir Technologies, Inc. Aktualna wersja to 3.12.4.

\subsubsection*{Sposób licencjonowania oraz cena produktu}

Wyłącznie na użytek służb bezpieczeństwa.

	\begin{tabular}{ | p{6cm}| p{6cm} | }
    	\hline
    	\textbf{Nazwa produktu} & \textbf{Cena}\\ \hline
	Palantir Gotham licencja wieczysta na 1 rdzeń serwera &   97 565 \pounds \\ \hline
    	Palantir Gotham miesięczna subskrybcja na 1 rdzeń serwera &  4 878 \pounds \\ \hline
  	\end{tabular}

\subsubsection*{Parametry techniczne i technologiczne}

Wymagania sprzętowe:
\begin{itemize}
	\item Minimalne:
		\begin{itemize}
		\item Intel Core 4 i7-2600 @ 3.4GHz+
    		\item nVidia lub ATI z 64 MB RAM, PCIe (8x)
		\item Ethernet
		\item monitor 1280x1024, 32-bit True Color
		\item 4GB RAM 
		\item 128GB HDD 
		\end{itemize}
	\item Zalecane: 
		\begin{itemize}
		\item Intel Core 4 i7-2600 @ 3.4GHz+
    		\item nVidia lub ATI z 1 GB RAM, PCIe (16x)
		\item Fast Ethernet
		\item kilka monitorów 1280x1024, 32-bit True Color
		\item 12GB lub więcej RAM 
		\item 200 GB lub więcej miejsca na dysku
		\end{itemize}
\end{itemize}


\subsection{CrimeStat}

CrimeStat to program służący do analizy danych przestrzennych na temat popełnionych przestępstw. Mam na celu zapewnienie dodatkowych narzędzi statystycznych, które ułatwią pracę analitykom śledczym organów ścigania. Program pobiera dane geoinformacyjne pochodzące od programów typu \emph{GIS} (np. ArcGIS), a następnie produkuje dane wyjściowe wzbogacone o szereg danych statystycznych, które mogą zostać na powrót wyświetlone przez oprogramowanie typu \emph{GIS}. 

\subsubsection*{Zakres dostępnych funkcji}

\begin{itemize}
	\item Mnogość funkcji agregujących dane oraz wyliczających statystyki
	\item Tworzenie wielu warstw danych statystycznych na mapach
	\item Identyfikacja schematu działania przestępców
	\item Wykrywanie obszarów zagrożonych kolejnym atakiem
	\item Optymalizacja tras patrolów służb mundurowych
\end{itemize}

\subsubsection*{Intuicyjność interfejsu oraz łatwość obsługi}

Dzięki wbudowanemu interfejsowi graficznemu prezentacja zgromadzonych wyników jest bardzo czytelna. Mnogość dostępnych funkcji pozwala w dużym stopniu na dostosowanie aplikacji do własnych potrzeb.

\subsubsection*{Stan prac na produktem}

Produkt został stworzony przez organizację \emph{Ned Levine and Associates} pod kierownictwem dr Ned'a Levine'a dzięki funduszom pochodzącym z Narodowego Instytutu Sprawiedliwości Stanów Zjednoczonych. Ostatnia aktualizacja oprogramowania do wersji \emph{3.3} miała miejsce w 31 lipca 2010 roku. Obecnie produkt nie jest rozwijany.

\subsubsection*{Sposób licencjonowania oraz cena produktu}

Oprogramowanie CrimeStat jest własnością \emph{Ned Levine and Associates} i jest przeznaczone dla instytucji wymiaru sprawiedliwości oraz organów ścigania. Może być darmowo rozpowszechniane przez instytucje naukowe do celów badawczych i edukacyjnych, ale nie może być sprzedawane. Objęte jest licencją \emph{firmware}.

\subsubsection*{Parametry techniczne i technologiczne}

Wymagania sprzętowe:
\begin{itemize}
	\item Minimalne:
		\begin{itemize}
			\item procesor taktowany częstotliwością 800 MHz
    		\item 256 MB pamięci RAM
		\end{itemize}
	\item Zalecane: 
		\begin{itemize}
			\item procesor taktowany częstotliwością 1.6 GHz
    		\item 1 GB pamięci RAM
		\end{itemize}
\end{itemize}

\raggedright{Wymagane oprogramowanie:}
\begin{itemize}
	\item System operacyjny:  Windows 2000/XP/Vista
\end{itemize}

\newpage
\section{Bibliografia}

\begin{itemize}
	\item Żabińska M.: Wykłady w ramach przedmiotu Projektowanie Systemów Informatycznych wygłoszone dla studentów II roku kierunku Informatyka Wydz. IEiT AGH w roku akademickim 2012/2013.
	\item Analyst’s Notebook: \url{http://www.acsys.com.pl/index.php?action=ANALYST}
	\item ArcGIS: \url{http://www.esriuk.com/software/arcgis}
	\item Maltego: \url{https://www.paterva.com/web6/products/maltego.php}
	\item Palantir: \url{http://www.palantir.com/}
	\item CrimeStat: \url{http://www.icpsr.umich.edu/CrimeStat/}
\end{itemize}

\section{Wykorzystane narzędzia}

\begin{itemize}
	\item Google Docs
	\item \emph{LaTeX}
\end{itemize}

\newpage
\section{Raport z działań}

Wykaz zadań wykonanych przez poszczególnych członków grupy projektowej:
\begin{itemize}
	\item \emph Krzysztof \textsc{Trzepla} 
		\begin{enumerate}
			\item Wstępna analiza aplikacji: Analyst's Notebook, Maltego, CrimeStat
		\end{enumerate}
	\item \emph Michał \textsc{Bigaj} 
		\begin{enumerate}
			\item Wstępna analiza aplikacji: ArcGIS, Palantir
		\end{enumerate}
\end{itemize}

\end{document}
